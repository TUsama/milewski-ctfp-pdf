% !TEX root = ../../ctfp-print.tex

\begin{quote}
  一直以来,我都想写一本介绍范畴论的书,而书的主要面向受众为程序员。
注意,我在这专指的是程序员以及软件工程师这类群体,计算机科学家这种不在此列

听起来像是一个大工程,对吧?其实,我本人对此也是颇为诚惶诚恐。我清楚地
意识到,在计算机科学与实际开发之间有着巨大的鸿沟,因为我本人对此二者均有涉猎。
我对于解释事物至始至终都抱有极大的热情。
也正因此,我深深地敬佩理查德·费曼(Richard Feynman)先生,
他真正做到了化繁为简,将复杂事物浓缩与简单的解释之中。我自是不如费曼先生,但我仍会尽全力写好这本书。
我希望能以此书为契机,激发读者对范畴论的兴趣。
所以我会将本序章公开发布,希望各位读者能在阅读后积极讨论并给予我反馈。
\footnote{
我也制作了一份本书的有声教学资料,点击以下链接即可查看:
\href{https://goo.gl/GT2UWU}{https://goo.gl/GT2UWU}(抑或在 YouTube 上搜索“bartosz milewski category theory”)
}。
\end{quote}

\lettrine[lhang=0.17]{我}{会在}接下来的几段介绍中解决如下的两个
问题:你为什么需要读这本书?范畴论是如此的抽象,你
是否需要会花掉大量的业余时间才能学通?我的答案是,你确实应该阅读本书,
而且学习范畴论也不会耗费你大把的时间。

我可不是乱下结论,这是我观察得来的。首先,范畴论乃是编程思想的
宝库,其中埋藏着诸多颇为有用的思想。可以说,使用
 Haskell 的程序员就是探索该宝库的先锋。随着编程语言的
发展,范畴论中的诸多概念也开始渗透到了其他语言之中,
只是效率比较低。我们得为这一进程加加速。

那么学习耗时何如?须知,数学中也有不同的分支,
分支也是挑人的。你也许一听到微积分与数论就直犯恶心,
但这并不代表范畴论也是如此。我会尽可能地向你
证明,范畴论其实是一门特别适合程序员的数学分支,它能与编程思维相得益彰。
这是由范畴论本身的特质决定的:范畴论处理的是结构,
而非一个个独立而特殊的事物。而且它处理的那种结构,恰好就是
程序中能够一块块组合起来的结构。

组合乃是范畴论的基础,是范畴定义中的一部分。
我会在本书中向你证明,组合
才是编程的本质。早在各路资深软件工程师创造出
子例程(subroutine)这一概念前,我们便已经走在“组合”这一道路上了。
结构化程序设计的出现如同一道春风,将编程带入了新的篇章,
原先分散独立的代码块得以相互组合在一起。
随之而来的便是面向对象编程的兴起,这一弄潮儿已是全然关于对象的组合了。
函数式编程从来都不只是与函数的组合以及代数数据结构(algebraic data structures)
挂钩,它还让并发得以可能,这在其他的编程范式里无异于天方夜谭。


最后,我还有个秘密武器:一把能够精准切分无聊数学的菜刀。
我会用它来将数学“大卸八块”,以飨各位读者程序员。对于职业数学家而言,
设定前提时总是得万般小心,尽可能的让其直接而清晰;
证明命题则是得滴水不漏;最后还得将你的证明严格地构造成型。
这就是为何数学论文对于外行人而言如同天书。
我是一位专业出身的物理学家,在物理研究中
,我们通常会使用非形式推理来推进研究。例如,狄拉克提出的 δ 函数便是
其中之一,他用此函数来解决某些微分方程的问题。而数学家们总是以此来大加嘲笑。

很快他们便笑不出来了,因为 δ 函数开辟了微积分的一条新道路
——分布理论(distribution theory)。狄拉克的深远洞见可见一斑。


当然,将繁杂论证化为三言两语也是有代价的,有时我可能会写出一些
明显错误的东西。我会尽可能地确保,本书中非形式论证的
背后都有着一套扎实严谨的数学论证。我手头确实有一本
桑德斯·麦克莱恩(Saunders Mac Lane)的\emph{Categories for the Working Mathematician} 旧复印本,
就在我的床头柜上。

由于这本书的主要受众为\emph{程序员},所以我会将
范畴论中的所有主要概念用计算机代码重新展示出来。或许你已经意识到了,
函数式语言要更贴近数学的逻辑,大行其道的命令式语言则与数学迥异,
并且函数式语言的抽象能力更强。自然而然地便有人采取这样的一个观点:
不学 Haskell,就不算是入了函数式编程的门。
然而这样的观点无疑是不正确的,
因为它潜在地将范畴论与非函数式语言割裂开来。
所以,我会在本书中给出大量的 C++ 示例。当然,
C++的语法确实相当丑陋,冗杂的东西一多,整体模式便难以理清。你可能得
频繁的复制粘贴一些代码,而不是尝试从概念的角度理解其中的含义。你懂的,C++是这样的。



但是想完全摆脱开 Haskell 也不现实。你确实不用非得成为
专职 Haskell 的程序员,你应该将 Haskell 视为 C++ 中待实现
特性的集合。我上手 Haskell 时便是抱着这样的目的。
在学习的过程中,我发现它的语法非常简洁,类型系统
也十分的强大。这对我理解 C++ 中的模板、数据结构以及算法起到了
非常大的助力。但毕竟不是每位读者都对 Haskell 有所了解,
所以我会在书写的过程中逐步引入,并细致地解释相关的概念。


如果你是一位身经百战的程序员,你可能会发问:我码代码这么久以来都没有
接触过范畴论以及函数式编程,为什么我现在要开始学这些
新东西?你说得对,这是因为在现在的命令式语言中,
函数式风格的特性开始逐步出现,此乃大趋势。
哪怕是 Java,这一面向对象编程的大本营,
都给 lambda 开了绿灯。反观 C++,在每过几年便推出的新标准中,
也能找到跟上这一趋势的诸多痕迹,只是显得有些手忙脚乱罢了。
所有的这些变化都昭示了一点:一场巨大的变革即将到来
,或者用我们物理学家的行话来说,这是准备到了相变点。就像是
不断地煮水,最终会让水沸腾那样。而我们就是那锅水里的青蛙,
到底是继续在逐渐温暖的水中惬意游泳,还是开始为自己考虑退路,
选择权就在我们手中。

\begin{figure}[H]
  \centering
  \includegraphics[width=0.5\textwidth]{images/img_1299.jpg}
\end{figure}

\noindent
论及这一大趋势的推手,多核革新不得不提。
之前流行的编程范式中,例如面向对象编程,根本就是对并发、并行等
概念嗤之以鼻,认为并发并行只会给你的设计埋下暗雷,指不定什么时候
就会爆炸。面向对象编程的基础前提便是数据隐藏(Data Hiding)
,但当被隐藏的数据被多方共用以及修改时,便产生了数据争用(Data Race)。
为数据加上互斥锁(Mutex)是个不错的主意。然而,锁不能
使用组合,而且锁将数据隐藏起来后,会更容易发生死锁,
并且也会加剧调试的难度。


但是即便没有并发,愈发复杂的软件系统
也在不断地挑战命令式编程的极限。
一言以蔽之,
副作用变得越来越难以解决。而反观函数式编程,带有副作用的函数通常写起来方便快捷,
它们的副作用在原则上可以在名字以及注释中标出。
一个叫做 SetPassword 或者是 WriteFile 的函数明显就会
改变某些状态并产生副作用,这在我们的工作场景中十分多见。
单个有副作用的函数不值一提,但若是将多个
这样的函数层层嵌套堆叠,事情就会变得棘手起来。
注意,我不是在说
副作用本身棘手,而是当副作用藏在一个大的框架中,千丝万缕难以理清的时候,
事情才会棘手。随着框架的变大,命令式编程只会让副作用
越来越散乱无章。

硬件的日新月异以及软件的愈发复杂,迫使我们重新思考起一个问题:
编程的“地基”到底应该打在何处?我们就像是欧洲的哥特式大教堂的建造者们
那样,不断地提升工艺水平,追逐材料以及结构的极限。
在法国博韦有一座尚未完工的哥特式
\urlref{http://en.wikipedia.org/wiki/Beauvais_Cathedral}{教堂
},在教堂的一砖一瓦之上,你能够清晰地看到人类是如何探索极限的。
这座教堂在建造之初,建筑师们便誓要以此来打破之前
教堂高度以及采光的记录,但很不幸,一系列的垮塌事故葬送了这一梦想。现今教堂中
搭建了众多铁棒以及木支架,用来防止整座建筑崩塌,但很明显,根子上的错误无法通过打补丁纠正。
从现代的视角来看,这一教堂的建造既无现代材料科学的指导,
也无计算机建模,更不用说有限元分析或是高等的数学与物理知识,
但却还是搭建出了如此之多的哥特式结构。不得不说,此乃奇迹。
而遥望未来,我们的子孙后代们面对着由我们编写出
的复杂操作系统、网页服务端以及互联网基础设施,我希望
他们也能由衷地赞叹这些编程技术。
不客气地说,他们理当赞叹,因为尽管我们的编程理论基础十分脆弱,却还是创造了如此辉煌的成就。
而当下,我们的使命就是打破束缚着编程技术的桎梏,创造出
更坚实的理论基础。

\begin{figure}
  \centering
  \includegraphics[totalheight=0.5\textheight]{images/beauvais_interior_supports.jpg}
  \caption{Ad hoc measures preventing the Beauvais cathedral from collapsing.}
\end{figure}